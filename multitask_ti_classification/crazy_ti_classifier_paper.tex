\documentclass[12pt]{article}
\usepackage{geometry}
\geometry{margin=1in}
\usepackage{amsmath, amssymb}
\usepackage{graphicx}
\usepackage{hyperref}
\usepackage{authblk}
\usepackage{setspace}
\usepackage{enumitem}
\usepackage{natbib}
\usepackage{lineno}
\usepackage{color}
\usepackage{caption}
\usepackage{float}
\usepackage{booktabs}
\usepackage{url}
\usepackage{lipsum}

\title{A Multimodal, Ensemble, and Transformer-Fusion Machine Learning Pipeline for Topological Quantum Material Classification}

\author[1]{Your Name}
\author[2]{Collaborator 1}
\author[3]{Collaborator 2}
\affil[1]{Department of Physics, Your University, Your City, Country}
\affil[2]{Department of Materials Science, Collaborator University, City, Country}
\affil[3]{Department of Computer Science, Collaborator University, City, Country}
\date{\today}

\begin{document}
\maketitle

\begin{abstract}
Prediction and discovery of new topological materials is a central challenge in quantum materials research. We present a comprehensive, aggressive machine learning pipeline that leverages multimodal data fusion, advanced graph neural networks, transformer-based cross-modal attention, self-supervised pretraining, and ensemble learning to maximize classification accuracy for topological insulators, semimetals, and trivial materials. Our approach integrates crystal and k-space graphs, scalar and decomposition features, and spectral graph features using a multi-branch transformer fusion model with deep residual connections. The pipeline incorporates state-of-the-art data augmentations, hyperparameter optimization, and automated analysis, providing a robust and extensible framework for high-throughput topological material discovery. (Results and quantitative benchmarks will be presented in a future version.)
\end{abstract}

\section{Introduction}
Recent advances in machine learning have enabled rapid progress in materials discovery, particularly for quantum materials with complex topological properties. Traditional ab initio calculations are computationally expensive, motivating the development of data-driven models that can predict material classes orders of magnitude faster. While previous works have explored graph neural networks (GNNs) and topological data analysis for material property prediction, robust frameworks for accurate topological class prediction from atomistic and electronic structure data remain limited. Here, we introduce a fundamentally new pipeline that aggressively combines multimodal data, advanced GNNs, transformer-based fusion, self-supervised learning, and ensemble methods to set a new standard for topological material classification.

\section{Methods}
\subsection{Overview of the Pipeline}
Our pipeline consists of the following major components:
\begin{enumerate}[label=\arabic*.]
    \item \textbf{Multimodal Data Integration:} We combine crystal graphs, k-space graphs, scalar features, decomposition features, and GPU-accelerated spectral graph features for each material.
    \item \textbf{Multi-Branch Transformer Fusion Model:} Each modality is encoded by a dedicated neural network (e.g., CGCNN, k-space GNN, spectral encoder), and the resulting embeddings are fused using a transformer-based cross-modal attention module with deep residual connections.
    \item \textbf{Advanced Data Augmentation:} Mixup, CutMix, and feature masking are applied during training to improve generalization and robustness.
    \item \textbf{Self-Supervised Pretraining:} GNN encoders are pretrained using node/edge prediction and contrastive learning tasks to leverage unlabeled data.
    \item \textbf{Ensemble Learning:} Multiple model variants are trained and combined using soft/hard/weighted voting to maximize predictive performance.
    \item \textbf{Hyperparameter Optimization:} Optuna-based search is used to tune architecture and training parameters.
    \item \textbf{Automated Analysis:} The pipeline includes t-SNE/UMAP visualization, attention heatmaps, error analysis, and feature importance reporting.
    \item \textbf{Baselines:} XGBoost, LightGBM, Random Forest, and Logistic Regression models are trained on extracted features for comparison.
\end{enumerate}

\subsection{Multimodal Data Representation}
\begin{itemize}
    \item \textbf{Crystal Graphs:} Each material's atomic structure is represented as a graph, with nodes encoding atom types and features, and edges encoding chemical bonds and distances. Graphs are constructed using Voronoi tessellation and covalent radii criteria.
    \item \textbf{K-space Graphs:} Electronic structure information is encoded as graphs in momentum space, capturing band connectivity and symmetry features.
    \item \textbf{Scalar and Decomposition Features:} Scalar descriptors (e.g., bandgap, symmetry indicators) and decomposition features (e.g., irreducible representations) are included as additional input vectors.
    \item \textbf{Spectral Graph Features:} GPU-accelerated spectral encoders compute Laplacian eigenvalues and related invariants for each graph, providing topological signatures.
\end{itemize}

\subsection{Model Architecture}
\begin{itemize}
    \item \textbf{Encoders:} Each modality is processed by a dedicated encoder: CGCNN for crystal graphs, GAT/GraphSAGE/GCN/TransformerConv for k-space graphs, MLPs for scalar/decomposition features, and a GPU spectral encoder for spectral features.
    \item \textbf{Transformer Fusion:} The encoded features are fused using a multi-head cross-modal transformer with deep residual connections, enabling complex interactions between modalities.
    \item \textbf{Classifier:} The fused representation is passed through a deep MLP with batch normalization, dropout, and focal loss for final prediction.
\end{itemize}

\subsection{Training and Augmentation}
\begin{itemize}
    \item \textbf{Data Augmentation:} Mixup, CutMix, and feature masking are applied to both graph and vector features during training.
    \item \textbf{Loss and Optimization:} Focal loss is used to address class imbalance. Training uses AdamW optimizer with cosine annealing and warm restarts.
    \item \textbf{Self-Supervised Pretraining:} GNN encoders are pretrained on node/edge prediction and contrastive tasks before supervised fine-tuning.
    \item \textbf{Ensemble Training:} Multiple model variants are trained with different seeds and hyperparameters, and their predictions are ensembled.
    \item \textbf{Hyperparameter Search:} Optuna is used to optimize architecture and training parameters.
\end{itemize}

\subsection{Automated Analysis and Baselines}
\begin{itemize}
    \item \textbf{Visualization:} t-SNE and UMAP are used to visualize learned representations. Attention heatmaps and feature importance plots are generated automatically.
    \item \textbf{Error Analysis:} Misclassified samples are analyzed and reported.
    \item \textbf{Baselines:} XGBoost, LightGBM, Random Forest, and Logistic Regression are trained on extracted features for comparison.
\end{itemize}

\subsection{Implementation Details}
The pipeline is implemented in Python using PyTorch, PyTorch Geometric, Optuna, and scikit-learn. All graph operations are GPU-accelerated. Dummy data loaders are used for initial testing; real data loaders are provided for production runs. The codebase is modular and extensible, supporting easy addition of new modalities or models.

% -------------------
% Results Section (Placeholder)
% -------------------
\section{Results}
% TODO: Insert quantitative results, tables, and figures here.
% \begin{figure}[H]
%     \centering
%     \includegraphics[width=0.7\textwidth]{fig1_placeholder.png}
%     \caption{Schematic of the multimodal transformer fusion model.}
%     \label{fig:model}
% \end{figure}

\section{Discussion}
Our aggressive multimodal pipeline demonstrates the power of combining graph-based, topological, and scalar features using advanced neural architectures and training strategies. The transformer fusion approach enables rich cross-modal interactions, while self-supervised pretraining and ensemble learning maximize robustness and accuracy. Automated analysis tools provide deep insight into model behavior and failure modes. We anticipate that this pipeline will accelerate the discovery of new topological materials and can be readily extended to other materials informatics tasks.

\section{Methods}
\subsection{Dataset}
The dataset is constructed from the Topological Quantum Chemistry database and related sources, comprising tens of thousands of materials labeled as trivial, semimetal, or topological insulator. Crystal structures, k-space data, and scalar descriptors are extracted and preprocessed. Data cleaning, conversion, and augmentation steps are automated in the pipeline.

\subsection{Implementation Details}
All models are implemented in PyTorch and PyTorch Geometric. Training is performed on modern GPUs. Hyperparameter search is managed by Optuna. Automated scripts are provided for data preprocessing, training, evaluation, and analysis. The codebase is open source and modular.

\section{Data Availability}
The data used in this study is available from the Topological Quantum Chemistry database (\url{https://www.topologicalquantumchemistry.org/}) and related sources. Processed datasets and scripts will be made available upon publication.

\section{Code Availability}
The full codebase, including model definitions, training scripts, and analysis tools, will be released at: \url{https://github.com/yourusername/crazy-ti-classifier}

\section{Acknowledgements}
We thank our collaborators and funding agencies for their support. Computational resources were provided by [Your Institution/Consortium].

\section{Author Contributions}
A.B. conceived the project. A.B. and C.D. designed the pipeline. E.F. implemented the code. All authors contributed to writing and editing the manuscript.

\section{Competing Interests}
The authors declare no competing interests.

\section{References}
% Example references (update as needed)
\begin{thebibliography}{99}
\bibitem{Xie2018} Xie, T. \& Grossman, J. C. Crystal graph convolutional neural networks for an accurate and interpretable prediction of material properties. \textit{Phys. Rev. Lett.} \textbf{120}, 145301 (2018).
\bibitem{Jiang2021} Jiang, Y. et al. Topological representations of crystalline compounds for the machine-learning prediction of materials properties. \textit{npj Comput. Mater.} \textbf{7}, 1--8 (2021).
\bibitem{Fey2019} Fey, M. \& Lenssen, J. E. Fast graph representation learning with pytorch geometric. arXiv:1903.02428 (2019).
\bibitem{Kingma2015} Kingma, D. P. \& Ba, J. Adam: A method for stochastic optimization. In \textit{3rd International Conference on Learning Representations, ICLR} (2015).
\bibitem{Pedregosa2011} Pedregosa, F. et al. Scikit-learn: Machine learning in Python. \textit{J. Mach. Learn. Res.} \textbf{12}, 2825--2830 (2011).
% Add more references as needed
\end{thebibliography}

\end{document} 